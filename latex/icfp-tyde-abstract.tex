\documentclass[sigplan,nonacm,screen,review,timestamp]{acmart}

% https://icfp22.sigplan.org/home/tyde-2022#Call-for-Papers

\usepackage[utf8]{inputenc}
\usepackage{amsmath}
\usepackage{amsthm}
\usepackage{stmaryrd}
\usepackage{listings}
\usepackage{xcolor}
\usepackage{xspace}
\usepackage{natbib}
\usepackage{todonotes} % TODO: remove at the end

\citestyle{acmauthoryear}

\newcommand{\Draft}[1]{{\color{gray} - #1}}
%\newcommand{\Todo}[1]{}
\newcommand{\Todo}[1]{\todo[inline,backgroundcolor=orange!30]{TODO: #1}}

\newcommand{\I}[1]{\texttt{#1}\xspace}
\newcommand{\K}[1]{\textbf{\texttt{#1}}\xspace}
\newcommand{\Interpret}[1]{\llbracket #1 \rrbracket\xspace}
\newcommand{\Floor}[1]{\lfloor #1 \rfloor\xspace}
\newcommand{\Existential}[2]{\Sigma[ #1 \in #2 ]\xspace}

\title{Provingly Correct Optimisations on Intrinsically Typed Expressions}
\subtitle{Extended Abstract}

% TODO: does this make sense?
% \keywords{dependent types, agda, compiler, type safety, program optimisation}

\author{Matthias Heinzel}
\affiliation{%
  \institution{Utrecht University}
  \city{Utrecht}
  \country{Netherlands}}
\email{m.h.heinzel@students.uu.nl}

\acmConference[TyDe’22]{Workshop on Type-Driven Development}{September 11}{Ljubljana, Slovenia}

\date{\today}

\begin{document}

\maketitle

\section{Introduction}

When writing a compiler for a functional programming language,
an important consideration is the treatment of binders and variables.
%\Draft{many possible representations (string, de Bruijn)}
%\Draft{naive implementation is not scope-safe, therefore partial}
A well-known technique when using dependently typed programming languages such as Agda
\cite{norell2007agda}
is to define an intrinsically typed syntax tree \cite{augustsson1999intrinsic}.
Expressions are scope- and type-safe by construction and admit a total evaluation function.
This construction has featured in several papers, exploring
basic operations like renaming and substitution
\cite{allais2018universe}
as well as compilation to different target languages
\cite[supplemental material]{pickard2021calculating}.

Optimisations play an important role in compilers, but
establishing their correctness is often not trivial,
with ample opportunity for mistakes.
However, there has been little focus on performing optimisations on intrinsically typed programs.
%
In this setting, program \emph{analysis} not only needs to identify optimisation opportunities,
but provide a proof witness that the optimisation is safe,
e.g. that some dead code is indeed not used.
For \emph{transformations} on intrinsically typed programs,
the programmer can rely on the compiler to check the relevant invariants,
but it can be cumbersome to make it sufficiently clear that type- and scope-safety are preserved,
especially when manipulating binders and variables.

Depending on the features of a language, there is a different,
potentially large number of relevant optimisations.
Therefore, we are limiting ourselves to a simple expression language
and widely applicable optimisations that only require bindings and variables to be present.
% WOUTER - how about something like - in this abstract we start to
% explore one optimization for a simple language. You don't need to
% mention that there are lots of different languages/optimizations out
% there -- that's kind of obvious -- but instead focus on what you
% want to tell here.

% WOUTER - you may want to cite Graham's (draft) paper - it's as easy
% as 1,2,3 that argues that studying semantics is best done in a
% setting that is as simple as possible.

% \Todo{Not sure how to plug the Nielsen book here, maybe just leave it out}
% list interesting optimisations?

Since our work is still in progress,
we will mainly present a specific optimisation, \emph{dead binding elimination}.
It is implemented by first annotating expressions with variable usage information
and then removing bindings that turn out to be unused.
We further prove that the optimisation is semantics-preserving.


\section{Dead Binding Elimination}

\subsection{Intrinsically Typed Expressions with Binders}

We define a simple typed expression language with let-bindings,
variables, primitive values, and addition as an example of a binary operator.
Since the optimisations we are interested in relate to variables and binders only,
the choice of possible values and additional primitive operations on them is mostly arbitrary.

\begin{align*}
  P, Q ::= v
  \ \big|\  P + Q
  \ \big|\  \textbf{let } x = P \textbf{ in } Q
  \ \big|\  x
\end{align*}

In Agda, the type of expressions $\I{Expr}$ is indexed by its return type ($\tau : \I{U}$)
and context ($\Gamma : \I{Ctx}$).
% WOUTER - in this grammar you only have Nats, so indexing with the number of
% variables would suffice... Perhaps adding some bool ops make sense?

Each free variable is a de Bruijn index into the context and acts as a proof that
the context contains an element of the matching type.
We can see how the context changes when introducing a new binding:

\begin{align*}
  \K{data}\ \I{Expr}& : (\Gamma : Ctx) (\tau : \I{U}) \to \I{Set}\ \K{where} \\
  \I{Let}
    :\ &\I{Expr}\ \Gamma\ \sigma \to
        \I{Expr}\ (\sigma :: \Gamma)\ \tau \to
        \I{Expr}\ \Gamma\ \tau  \\
    \ldots\ &
\end{align*}

This allows the definition of a total evaluator
using a matching environment:

\begin{align*}
  \I{eval} &: \I{Expr}\ \Gamma\ \tau \to \I{Env}\ \Gamma \to \Interpret{\tau}
\end{align*}

\subsection{Sub-contexts}

\Todo{cite \emph{A correct-by-construction conversion to combinators}? It's quite similar.}

Note that an expression is not forced to make use of the whole context to which it has access.
Specifically, a let-binding introduces a new element into the context, but it might never be used.
To reason about the \emph{sub-contexts} that are live (actually used),
we use \emph{order-preserving embeddings} (OPE) \cite{chapman2009type}.
For each element of a context, a sub-context specifies whether to keep it or not.

\begin{align*}
  &\K{data}\ \I{Subset} : \I{Ctx} \to \I{Set}\ \K{where}          \\
  &\ \ \ \ \I{Empty} : \I{Subset}\ []                                    \\
  &\ \ \ \ \I{Drop}  : \I{Subset}\ \Gamma \to \I{Subset}\ (\tau :: \Gamma)  \\
  &\ \ \ \ \I{Keep}  : \I{Subset}\ \Gamma \to \I{Subset}\ (\tau :: \Gamma)
\end{align*}

Such a sub-context describes a context themselves,
given by a function $\Floor{\_} : \I{Subset}\ \Gamma \to \I{Ctx}$,
but it contains more information than that.
For example, the witnesses of a binary relation $\subseteq$ on sub-contexts are unique,
as opposed to working on contexts directly, e.g. $[\I{INT}] \subseteq [\I{INT}, \I{INT}]$.
% WOUTER - I'm not sure I understand this remark.. you haven't defined
% subseteq, which makes it a bit confusing.  Perhaps it suffices to
% note that these Subsets determine a unique sub-context?

From now on, we will only consider expressions
$\I{Expr}\ \Floor{\Delta}\ \tau$ in some sub-context.
Initially, we take $\Delta = \I{all}\ \Gamma : \I{Subset}\ \Gamma$,
the complete sub-context of the original context.

\subsection{Live Variable Analysis}

Now we can annotate each expression with its \emph{live variables},
the sub-context $\Delta' \subseteq \Delta$ that is really used.
To that end, we define annotated expressions $\I{LiveExpr}\ \Delta\ \Delta'\ \tau$.
While $\Delta$ is treated as $\Gamma$ before, $\Delta'$ now only contains live variables,
starting with a singleton sub-context at the variable usage sites.

\begin{align*}
  \K{data}\ \I{LiveExpr}& : (\Delta\ \Delta' : \I{Subset}\ \Gamma) (\tau : \I{U}) \to \I{Set}\ \K{where} \\
  \I{Let}
    :\ &\I{LiveExpr}\ \Delta\ \Delta_1\ \sigma \to  \\
       &\I{LiveExpr}\ (\I{Keep}\ \Delta)\ \Delta_2\ \tau \to  \\
       &\I{LiveExpr}\ \Delta\ (\Delta_1 \cup \I{pop}\ \Delta_2)\ \tau  \\
    \ldots\ &
\end{align*}

To create such annotated expressions, we need to perform
some static analysis of our source programs.
The function \I{analyse} computes the live sub-context $\Delta'$
together with a matching annotated expression.
The only requirement we have for it is that we can forget the annotations again,
with $\I{forget} \circ \I{analyse} \equiv \I{id}$.

\begin{align*}
  \I{analyse}
    &:  \I{Expr}\ \Floor{\Delta}\ \tau
    \to \Existential{\Delta'}{\I{Subset}\ \Gamma}\ \I{LiveExpr}\ \Delta\ \Delta'\ \tau \\
  \I{forget}
    &:  \I{LiveExpr}\ \Delta\ \Delta'\ \tau
    \to \I{Expr}\ \Floor{\Delta}\ \tau
\end{align*}

\Todo{
  Maybe add a note that \I{LiveExpr} is overspecified.
  Instead of $\Delta_1 \cup \Delta_2$ we could have any $\Delta'$ containing $\Delta_1$ and $\Delta_2$.
}

\subsection{Transformation}

Note that we can evaluate $\I{LiveExpr}$ directly, with the main difference
%WOUTER - scrap 'with' in the line above?
that in the $\I{Let}$-case we match on $\Delta_2$ to distinguish whether the bound variable is live.
If it is not, we directly evaluate the body, ignoring the bound declaration.
Another important detail is that evaluation works under any environment containing (at least) the live context.

\begin{align*}
  &\I{evalLive} : \\
  &\ \ \ \ \I{LiveExpr}\ \Delta\ \Delta'\ \tau \to  
           \I{Env}\ \Floor{\Delta_u} \to  
           .(\Delta' \subseteq \Delta_u) \to  
           \Interpret{\tau}
\end{align*}

This \emph{optimised semantics} shows that we can do a similar program transformation
and will be useful in its correctness proof.
% WOUTER - does not contain more that -> simply maps each constructor from LiveExpr to its
% counterpart in Expr.
The implementation simply maps each constructor to its counterpart in \I{Expr},
with some renaming
(e.g. from $\Floor{\Delta_1}$ to $\Floor{\Delta_1 \cup \Delta_2}$).
and the abovementioned case distinction.

\begin{align*}
  &\I{dbe} :  \I{LiveExpr}\ \Delta\ \Delta'\ \tau \to \I{Expr}\ \Floor{\Delta'}\ \tau  \\
  &\I{dbe}\ (\I{Let}\ \{\Delta_1\}\ \{\I{Drop}\ \Delta_2\}\ e_1\ e_2) =
      \I{injExpr}_2\ \Delta_1\ \Delta_2\ (\I{dbe}\ e_2)  \\
  &\I{dbe}\ \ldots
\end{align*}

As opposed to \I{forget}, which returns to the original context,
here we remove unused variables, only keeping $\Floor{\Delta'}$.

\subsection{Correctness}

We want to show that dead binding elimination preserves semantics:
$\I{eval} \circ \I{dbe} \circ \I{analyse} \equiv \I{eval}$.
Since we know that $\I{forget} \circ \I{analyse} \equiv \I{id}$,
it is sufficient to show the following:

\begin{align*}
  \I{eval} \circ \I{dbe} \equiv \I{eval} \circ \I{forget}
\end{align*}

The proof gets simpler if we split it up using the optimised semantics.

\begin{align*}
  \I{eval} \circ \I{dbe} \equiv \I{evalLive} \equiv \I{eval} \circ \I{forget}
\end{align*}

\Todo{details about proof statement, generalisation of environment used etc.}

Both proofs work inductively on the expression, with most cases being a straight-forward congruence.
The interesting case is \I{Let}, \ldots

\Todo{continue}

\subsection{Iterating the Optimisation}

% When we remove a binding that contains the only occurrences of some other variable,
% we can run the optimisation again to remove that binding as well.
% This is common among optimisations, which then can be iteratively applied until a fixpoint is reached.
% Sometimes (as with \emph{strong live variable analysis} here)
% it is possible to achieve the same result in a single pass,
% but at the cost of additional complexity.

A binding that is removed can contain the only occurrences of some other variable.
This makes another binding dead, allowing further optimisation when running the algorithm again.
While in our simple setting all these bindings could be identified in a single pass
using \emph{strong live variable analysis},
in general it can be useful to simply iterate the optimisation until a fixpoint is reached.

% NOTE: number of iterations not statically known
Such an iteration is not structurally recursive, so Agda's termination checker needs our help.
We observe that the algorithm must terminate
since the number of bindings decreases with each iteration (but the last) and cannot become negative.
This is the same as the ascending chain condition in program analysis literature
\cite{nielsen1999analysis}.
To convince the termination checker, we use \emph{well-founded recursion} \cite{bove2016recursion}
on the number of bindings.

The correctness follows directly from the correctness of each individual iteration step.

\section{Preliminary Results}

The implementation and correctness proof of dead binding elimination are complete,
the Agda source code is available online
\footnote{\url{https://git.science.uu.nl/m.h.heinzel/correct-optimisations/-/tree/tyde}}.
One interesting observation is that the correctness proof does not rely on how
\I{analyse} computes the annotations.
At first, this does not seem particularly useful,
but for other optimisations the analysis might use complex, frequently changing heuristics to decide
which transformations are worth it.

We are currently extending the expression language
with $\lambda$-abstractions.
While some increase in complexity is necessary to eliminate applications of functions that do not use their argument,
the correctness proof seems to stay relatively simple.

We are further investigating additional binding-related transformations,
such as moving bindings up or down in the syntax tree.
Another interesting type of optimisation is avoidance of redundant computations
using \emph{available expression analysis}.
An example is \emph{common subexpression elimination},
where subexpressions get replaced by variables bound to equivalent declarations
(pre-existing or newly created).

Between the different optimisations,
we hope to discover common patterns and refine our approach,
providing useful strategies for performing optimisations in intrinsically typed compilers.
%WOUTER - or more generally, manipulating well-typed well-scoped syntax trees?

\bibliographystyle{ACM-Reference-Format}
\bibliography{bibliography}{}

\end{document}
