\documentclass[11pt,a4paper]{article}

\usepackage[utf8]{inputenc}
\usepackage{amsmath}
\usepackage{amsthm}
\usepackage{stmaryrd}
\usepackage{listings}
\usepackage[pdftex]{xcolor}
\usepackage{xspace}
\usepackage{agda}
\usepackage{cite}
\usepackage{hyperref}
\usepackage{todonotes}

%\newcommand{\Draft}[1]{}
%\newcommand{\Todo}[1]{}
\newcommand{\Draft}[1]{\todo[inline,bordercolor=gray,textcolor=gray,backgroundcolor=gray!20]{#1}}
\newcommand{\Todo}[1]{\todo[inline,backgroundcolor=orange!30]{TODO: #1}}

\newcommand{\Floor}[1]{\lfloor #1 \rfloor\xspace}
\newcommand{\Existential}[2]{\Sigma[ #1 \in #2 ]\xspace}

% TODO: try instead: https://tex.stackexchange.com/questions/529907/how-to-set-the-font-which-agda-code-uses
\usepackage{newunicodechar}
\newunicodechar{∀}{\ensuremath{\mathnormal\forall}}
\newunicodechar{⊤}{\ensuremath{\mathnormal\top}}
\newunicodechar{⊥}{\ensuremath{\mathnormal\bot}}
\newunicodechar{∘}{\ensuremath{\mathnormal\circ}}
\newunicodechar{≡}{\ensuremath{\mathnormal\equiv}}
\newunicodechar{ᵇ}{\ensuremath{\mathnormal{}^b}}
\newunicodechar{ℕ}{\ensuremath{\mathnormal\mathbf{N}}}
\newunicodechar{∅}{\ensuremath{\mathnormal\emptyset}}
\newunicodechar{∈}{\ensuremath{\mathnormal\in}}
\newunicodechar{⊆}{\ensuremath{\mathnormal\subseteq}}
\newunicodechar{∪}{\ensuremath{\mathnormal\cup}}
\newunicodechar{∷}{\ensuremath{\mathnormal::}}
\newunicodechar{⟦}{\ensuremath{\mathnormal\llbracket}}
\newunicodechar{⟧}{\ensuremath{\mathnormal\rrbracket}}
\newunicodechar{⌊}{\ensuremath{\mathnormal\lfloor}}
\newunicodechar{⌋}{\ensuremath{\mathnormal\rfloor}}
\newunicodechar{Γ}{\ensuremath{\mathnormal\Gamma}}
\newunicodechar{σ}{\ensuremath{\mathnormal\sigma}}
\newunicodechar{τ}{\ensuremath{\mathnormal\tau}}
\newunicodechar{Δ}{\ensuremath{\mathnormal\Delta}}
\newunicodechar{Σ}{\ensuremath{\mathnormal\Sigma}}
\newunicodechar{₁}{\ensuremath{\mathnormal{}_1}}
\newunicodechar{₂}{\ensuremath{\mathnormal{}_2}}
\newunicodechar{ᵤ}{\ensuremath{\mathnormal{}_u}}

\include{generated/Lang}
\include{generated/Subset}
\include{generated/Recursion}
\include{generated/Live}

\newcommand{\AgdaFloor}[1]{\AgdaFunction{$\lfloor$}\ #1\ \AgdaFunction{$\rfloor$}}

\title{Provingly Correct Optimizations in an Intrinsically Typed Compiler\\
  \vspace{1cm}
  \large Experimentation Project Report}
\author{Matthias Heinzel (1632256)}
\date{\today}

\begin{document}

\maketitle
\tableofcontents
\pagebreak

\section{Introduction}
% MAYBE: Use "I" instead of "we"?
% MAYBE: make this more high-level and move citations to other sections?

When writing a compiler for a programming language,
an important consideration is the treatment of binders and variables.
A well-known technique when using dependently typed programming languages such as Agda
\cite{norell2007agda}
is to define an intrinsically typed syntax tree,
where expressions are scope- and type-safe by construction and admit a total evaluation function
\cite{augustsson1999intrinsic}.
This construction has featured in several papers, exploring
basic operations like renaming and substitution
\cite{allais2018universe}
as well as compilation to different target languages
\cite[supplemental material]{pickard2021calculating}.

Performing optimisations on intrinsically typed programs, on the other hand,
has not received as much attention.
However, optimisations play an important role in compilers
and establishing their correctness is often not trivial,
with ample opportunity for mistakes.
%
In this setting, program \emph{analysis} not only needs to identify optimisation opportunities,
but provide a proof witness that the optimisation is safe,
e.g. that some dead code is indeed not used.
For the \emph{transformation} of the intrinsically typed program,
the programmer can then rely on the compiler to check the relevant invariants,
but it can be cumbersome to make it sufficiently clear that type- and scope-safety are preserved,
especially when manipulating binders and variables.

As a step towards a more general treatment of optimisations of intrinsically typed programs,
we present an implementation of \emph{dead binding elimination} for a simple language.
It first annotates expressions with variable usage information
and then removes bindings that turn out to be unused.
We further prove that the optimisation is semantics-preserving.
The Agda source code is available online at
\url{https://git.science.uu.nl/m.h.heinzel/correct-optimisations}.

\section{A Simple Expression Language}

\subsection{Expressions with Bindings}

We define a simple, typed expression language with let-bindings,
variables, primitive values (integers and Booleans), and a few binary operators.
Since the optimisations we are interested in relate to variables and binders only,
the choice of possible values and additional primitive operations on them is mostly arbitrary.
Extending the language with further values and operators is trivial.

\begin{align*}
  P, Q ::=&\ v
  \\ \big|&\ P + Q
  \\ \big|&\ \textbf{let } x = P \textbf{ in } Q
  \\ \big|&\ x
\end{align*}

Expressions can be bound to a variable $x$ using the $\textbf{let}$ construction.
Note that this makes the language equivalent to a restricted version of the simply typed $\lambda$-calculus,
where $\lambda$-abstraction and application can only occur together as $(\lambda x. Q) P$.
Encapsulating this pattern as $\textbf{let } x = P \textbf{ in } Q$
simplifies parts of the analysis and
avoids the need for allowing functions as values.


\subsection{Representing Variables}

The syntax specified above treats variables as letters, or more generally strings.
To prevent complications with bindings of the same variable name shadowing each other
and to make equality of terms independent of the specific names chosen
(\emph{$\alpha$-equivalence}),
compilers often represent variables in a different way.
A popular choice are \emph{de Bruijn indices},
where each variable is represented by a natural number,
counting the number of bindings between variable occurence and its binding:
$0$ refers to the innermost binding, $1$ to the next-innermost etc.

Still, there might be \emph{free variables},
where the de Bruijn index is larger than the number of bindings it has access to
(\emph{in scope}).
If this happens unexpectedly during evaluation, an error is raised.
Similarly, the type of a bound expression might not match the expected type at the variable occurence
where it is referenced.
This makes the evaluation function partial;
it should only be called after validating type- and scope-safety.


\subsection{Intrinsically Typed Syntax Trees}

When implementing a compiler in a dependently typed programming language,
one does not need to accept partiality and the need for human vigilance.
With \emph{intrinsically typed syntax trees}, type- and scope-safety invariants
are specified on the type level and verified by the type checker.
We will demonstrate the approach in Agda and start by defining the types that terms can have.

\CodeLangTypes

To know if a variable occurence is valid, we must consider its \emph{context},
the bindings that are in scope.
With de Bruijn indices in an untyped setting, it would suffice to know the number of bindings in scope.
In a typed settings, we also need to know the type of each binding,
so we represent the context by a list of types: One for each binding in scope, from innermost to outermost.

\CodeLangCtx

During evaluation, each variable in scope has a value.
Together, we call these an \emph{environment} in a given context.

\CodeLangEnv

A variable then is an index into its context,
also guaranteeing that its type matches that of the binding.
Since variable $\AgdaDatatype{Ref}\ \sigma\ \Gamma$ acts as a proof that
the environment $\AgdaDatatype{Env}\ \Gamma$ contains an element of type $\sigma$,
variable lookup is total.

\CodeLangRef

Now we can define intrinsically typed expressions,
where an \AgdaDatatype{Expr} is indexed by both
its type ($\sigma : \AgdaDatatype{U}$)
and context ($\Gamma : \AgdaDatatype{Ctx}$).
We can see how the context changes when introducing a new binding
that is then available in the body of a \AgdaInductiveConstructor{Let}.

\CodeLangExpr

This allows the definition of a total evaluator
using an environment matching the expression's context.

\CodeLangSemantics


\section{Dead Binding Elimination}

Note that an expression is not forced to make use of the whole context to which it has access.
Specifically, a let-binding introduces a new element into the context, but it might never be used
in the body.
One commonly wants to identify such unused bindings so they can be removed from the program
\cite[Live Variable Analysis]{nielsen1999analysis}.
To that end, we annotate terms with the subset of their context that they actually use.


\subsection{Subsets of a Context}

A \emph{sub-context} is not just any context.
All its elements must come from the original context and we also require that their order is unchanged.
This notion is known as a \emph{thinning} or \emph{order-preserving embedding} (OPE) \cite{chapman2009type}
between the two contexts.
We combine the sub-context with its OPE in a single data type \AgdaDatatype{Subset},
which for each element of the context specifies whether to keep it or not.
The uniquely determined sub-context can be extracted using \AgdaFunction{$\Floor{\_}$}.

\CodeSubsetSubset

We can then define a subset-like order \AgdaFunction{$\subseteq$}:

\CodeSubsetOpSubseteq

Dealing with \AgdaDatatype{Subset} will simplify our task, especially the correctness proofs.
While an OPE itself could be used to express a similar relation between contexts,
its inhabitants are not unique.
For example, $\AgdaInductiveConstructor{[\ NAT\ ]}$
can be embedded into $\AgdaInductiveConstructor{[\ NAT\ ,\ NAT\ ]}$
either by keeping the first element or the second.
This would complicate equality proofs on terms including witnesses of the relation.
For similar reasons, it is unclear how we would want to define a union operator
$\AgdaFunction{$\_\cup\_$} : \AgdaDatatype{Ctx} \to \AgdaDatatype{Ctx} \to \AgdaDatatype{Ctx}$,
but it is trivial on \AgdaDatatype{Subset}s.

From now on, we will only consider expressions
$\AgdaDatatype{Expr}\ \AgdaFloor{\Delta}\ \sigma$ in some sub-context.
Initially, we take $\Delta = \AgdaFunction{all}\ \Gamma : \AgdaDatatype{Subset}\ \Gamma$,
the complete sub-context of the original context.


\subsection{Live Variable Analysis}

The next step is to annotate each expression with its \emph{live variables},
the sub-context $\Delta'$ that is really used.
To that end, we define annotated expressions
$\AgdaDatatype{LiveExpr}\ \Delta\ \Delta'\ \sigma$.
Note that $\Delta$ is treated as $\Gamma$ before,
conceptually accumulating all defined variables top-down.
On the other hand, $\Delta'$ accumulates used variables bottom-up,
starting with a singleton sub-context at the variable occurences.

\CodeLiveExpr

To create such annotated expressions, we need to perform
some static analysis of our source programs.
The function \AgdaFunction{analyse} computes the live sub-context $\Delta'$
together with a matching annotated expression.

% Horizontal bars in with-abstraction are broken...
% https://tex.stackexchange.com/questions/193216/vertical-bar-turning-horizontal
% Can be fixed in latex/generated/Live.tex with :%s/\\AgdaSymbol{|}/\\AgdaSymbol{\\textbar}/g
\CodeLiveAnalyse

The only requirement we have for it is that we can forget the annotations again.

\CodeLiveForgetSignature

\CodeLiveAnalysePreservesSignature


\subsection{Transformation}

It is useful to realise that we can evaluate \AgdaDatatype{LiveExpr} directly,
almost identically to the previous \AgdaFunction{eval} function.
The main difference is in the \AgdaInductiveConstructor{Let}-case,
where we match on $\Delta_2$ to distinguish whether the bound variable is live.
If it is not, we can directly evaluate the body, ignoring the bound declaration.

Another important detail is that evaluation works under an environment
for any sub-context $\Delta_u$
containing (at least) the live variables $\Delta'$.
Requiring an $\AgdaDatatype{Env}\ \AgdaFloor{\Delta}$ would force us
to provide a value even for unused variables.
Fixing it to exactly $\AgdaDatatype{Env}\ \AgdaFloor{\Delta'}$, however,
would make it necessary to project the environment for each recursive call,
since the subexpressions generally only use parts of the context.

\CodeLiveEvalLive

This \emph{optimised semantics} hints at a similar program transformation
and will be useful when proving it to be correct.
The transformation simply maps each constructor to its counterpart in \AgdaDatatype{Expr},
with some variable renaming
(e.g. from \AgdaFloor{$\Delta_1$} to \AgdaFloor{$\Delta_1$ \AgdaFunction{$\cup$} $\Delta_2$})
and the abovementioned case distinction,
where we can omit unused bindings.

\CodeLiveDbe
\CodeLiveRestrictedRefSignature

As opposed to the type signature of \AgdaFunction{forget},
which returns to the original context,
in \AgdaFunction{dbe} we drop unused variables, only keeping \AgdaFloor{$\Delta'$}.


\subsection{Correctness}

We want to show that dead binding elimination preserves semantics:
$\AgdaFunction{eval} \circ \AgdaFunction{dbe} \circ \AgdaFunction{analyse} \equiv \AgdaFunction{eval}$.
Since we know that
$\AgdaFunction{forget} \circ \AgdaFunction{analyse} \equiv \AgdaFunction{id}$,
it is sufficient to show
$\AgdaFunction{eval} \circ \AgdaFunction{dbe} \equiv \AgdaFunction{eval} \circ \AgdaFunction{forget}$.
The proof gets simpler if we split it up using the optimised semantics:

\begin{align*}
  \AgdaFunction{eval} \circ \AgdaFunction{dbe}
    \equiv \AgdaFunction{evalLive}
    \equiv \AgdaFunction{eval} \circ \AgdaFunction{forget}
\end{align*}

The actual proof statements are more involved,
since they quantify over the expression and environment used.
As foreshadowed in the definition of \AgdaFunction{evalLive}, the statements are also generalised
to evaluation under any $\AgdaDatatype{Env}\ \AgdaFloor{\Delta_u}$,
as long as it contains the live sub-context.
This gives us more flexibility when using the inductive hypothesis.

\CodeLiveDbeCorrectSignature
\CodeLiveEvalLiveCorrectSignature

Both proofs work inductively on the expression, with most cases being a straight-forward congruence.
The interesting one is again \AgdaInductiveConstructor{Let},
where we split cases on the variable being used or not
and need some auxiliary facts about evaluation, renaming and sub-contexts.
% MAYBE: more details?

An important detail is that the \AgdaFunction{$\subseteq$} proofs
are \emph{irrelevant arguments}, indicated by the preceding dot.
This is possible since their witnesses are unique
and makes it clear to the Agda type checker that e.g.
$\AgdaFunction{prjEnv}\ \Delta'\ \Delta\ H_1$ and
$\AgdaFunction{prjEnv}\ \Delta'\ \Delta\ H_2$ are equal.


\subsection{Iterating the Analysis}

The declaration of a binding that is removed
can contain the only occurrences of some other variable.
This makes another binding dead, allowing further optimisation when running the algorithm again.
While in our simple setting all these bindings could be identified in a single pass
using \emph{strong live variable analysis},
in general it can be useful to simply iterate the optimisation until a fixpoint is reached.

This approach is not structurally recursive and the number of iterations is not known upfront,
so Agda's termination checker needs our help.
We observe that the algorithm must terminate,
since the number of bindings decreases with each iteration
(except for the last) and cannot become negative.
This corresponds to the ascending chain condition in program analysis literature
\cite{nielsen1999analysis}.
To convince the termination checker, we use \emph{well-founded recursion} \cite{bove2016recursion}
on the number of bindings.
% MAYBE: explain well-founded recursion?

The overall correctness follows directly from the correctness of each individual iteration step.


\section{Further Work}

\subsection{Extending the Language}

Since our language only contains let-bindings,
it might be of interest to extend it with $\lambda$-abstractions
(forming a simply-typed $\lambda$-calculus).
Some increase in complexity seems necessary to eliminate applications of
functions that do not use their argument,
but we hope that our work is still largely applicable.
The problem gets more challenging when introducing recursive bindings.
Conversely, adding sum and product types might require more extensive bookkeeping,
but should not pose fundamental difficulties.


\subsection{Other Analyses}

There are several other binding-related transformations to explore,
such as moving bindings up or down in the syntax tree.
Another interesting type of optimisation is avoidance of redundant computations
using \emph{available expression analysis}.
An example is \emph{common subexpression elimination},
where subexpressions get replaced by variables bound to equivalent declarations
(pre-existing or newly created).

Ideally, further exploration will lead to the discovery of common patterns
and useful strategies for performing optimisations on intrinsically typed syntax trees.


\pagebreak
\bibliographystyle{plain}
\bibliography{bibliography}{}

\end{document}
