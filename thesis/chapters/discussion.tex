\chapter{Discussion}
\label{ch:discussion}
    \Outline{Quick summary: Did transformations, learned that liveness information is nice to have in syntax tree, this can be achieved by using co-de-Bruijn, at cost of some complexity}
    \Outline{Mismatch: Reordering bindings and thinnings. This extends to co-de-Bruijn, as it is built on thinnings. Alternatives: allowing re-orderings. thinnings as functions (also associativity for free). But challenges with that.}
    \Draft{
      Both thinning and reordering could be performed
      by a more general renaming operation,
      for example using a function
      |(forall sigma -> Ref sigma Delta -> Ref sigma Gamma)|
      mapping references in the source context
      to references in the target context.
      This representation however is more opaque
      and difficult to manipulate than thinnings.
    }

\section{Related Work}
\label{sec:related-work}
    \Outline{Heavily used work by McBride and Allais et~al.}
  \paragraph{Program analysis in Agda}
    \Outline{\cite{VanWijk2017DependentlyTypedMonotoneFrameworks}}
    \Fixme{Is this worth citing?}
  \paragraph{Liveness annotations}
    \Outline{Ornamentation? \cite{Dagand2014TransportingFunctionsAcrossOrnaments}?}
  \Fixme{More?}


\section{Further Work}
\label{sec:further-work}
    \Fixme{Copy some stuff from the proposal}

    \Draft{
      Shortcomings and unfinished work:
      \begin{itemize}
        \item finish proofs for let-sinking
        \item manipulating covers
        \item generic let-sinking (not that easy to specify desired behaviour: what to push into and what not?)
        \item semantics of syntax-generic transformations
        \item properly explore generic co-de-Bruijn to allow working with Semantics etc. (as mentioned by Allais)
      \end{itemize}
    }

\subsection{Extending the Language}
  \paragraph{Effects}
    \Outline{
      We used a very simple semantics, but it does not account for effects,
      including non-termination.
      Citations for how to reason about more complicated semantics? (operational, relations)
      \cite{Capretta2005GeneralRecursion,Danielsson2012PartialityMonad,McBride2015TuringCompletenessTotallyFree}
      Maybe something from Amal Ahmed?
    }
    \Outline{
      Also affects transformations we do.
      Bindings cannot be removed or reordered if the evaluation of their declaration
      is effectful.
    }
  \paragraph{Recursive bindings}
    \Outline{Non-termination, see effects. But also analysis gets more complicated.}

\subsection{Other Transformations}
  \paragraph{Common subexpression elimination}
  \paragraph{Let-floating}
    \Outline{Floating out of lambdas, aka full laziness transformation}
